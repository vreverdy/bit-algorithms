\documentclass{article}
\usepackage[utf8]{inputenc}
\usepackage[english]{babel}
\usepackage{graphicx}
\graphicspath{{images/}{../images/}} \usepackage[a4paper, total={6in, 8in}]{geometry} 
\usepackage{subfiles}
\usepackage{hyperref}
\usepackage{cleveref}
\usepackage{blindtext}
 
\hypersetup{
    colorlinks=true,
    linkcolor=blue,
    filecolor=magenta,      
    urlcolor=cyan,
}
\urlstyle{same}

\usepackage{listings}
\usepackage{xcolor}
\lstset { %
    language=C++,
    backgroundcolor=\color{black!5}, % set backgroundcolor
    basicstyle=\footnotesize,% basic font setting
    escapechar=|
}

\title{bit-algorithms Documentation}
\author{
    Vincent Reverdy
    \and
    Bryce Kille
    \and
    Collin Gress
}
 
\begin{document}
 
\maketitle

\setcounter{secnumdepth}{2}
\setcounter{tocdepth}{2}
\tableofcontents

\section{Introduction}
Here we will document algorithms and their progress. 
Sections will be divided according to how they are in the 
\href{https://en.cppreference.com/w/cpp/algorithm}{reference}. Each algorithm 
subsection will be introduced with a description of the algorithms behavior 
and signature, followed by the current implementation pseudocode in \texttt{bit}. 
Any other possible implementations may follow. At the end, a there can be 
a discussion of potential optimization, drawbacks, or bugs for each 
implementation. 
\section{Example operations}\label{ExOs}
\subsection{\texttt{max}}
\begin{lstlisting}
template< class T > 
constexpr const T& max( const T& a, const T& b ); |\label{max_sig_1}| (1)

template< class T, class Compare >
constexpr const T& max( const T& a, const T& b, Compare comp ); |\label{max_sig_2}| (2)

template< class T >
constexpr T max( std::initializer_list<T> ilist ); |\label{max_sig_3}| (3)

template< class T, class Compare >
constexpr T max( std::initializer_list<T> ilist, Compare comp ); |\label{max_sig_4}| (4)
\end{lstlisting}

\noindent Returns the greater of the given values.
\begin{itemize}
  \item[1-2)] Returns the greater of \texttt{a} and \texttt{b}.
  \item[3-4)] Returns the greatest of the values in intializer list \texttt{ilist}. 
\end{itemize}
The (1,3) versions use operator \texttt{<} to compare the values, the (2,4) versions use the given comparison function \texttt{comp}.

\subsubsection{Parameters}
Parameters go here. TODO find a better way to format these\\
\begin{tabular}{ l  l  l }
    \textbf{a, b} \qquad &- \qquad & \ the values to compare.\\
    \textbf{ilist} &-& initializer list with the values to compare.\\
    \textbf{comp} &-& 	comparison function object 
\end{tabular}

\subsubsection{Type requirements}

\subsubsection{Return value}

\subsubsection{Complexity}

\subsubsection{Implementation}

\subsubsection{Possible Implementations}

\subsubsection{Notes}
\section{Non-modifying sequence operations}\label{NmSOs} 
\section{Modifying sequence operations}\label{MSOs} 
 
\end{document}
